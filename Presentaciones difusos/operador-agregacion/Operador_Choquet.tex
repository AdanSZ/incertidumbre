\documentclass{beamer}

\usepackage[utf8]{inputenc}
\usepackage[spanish]{babel}
\usepackage{times}
\usepackage{graphicx}
\usepackage{amsmath}
\usepackage{amsthm}
\usepackage{amssymb} %También puede ser amsfonts
%\usepackage{amssymb}... para \restriction

\theoremstyle{definition}
\newtheorem{Def}{Definición}
\newtheorem{Prop}{Proposición}

\spanishdecimal{.}


\mode<presentation> {
  \usetheme{Madrid}
  \setbeamercovered{transparent}
}

\title[Integral de Choquet] %
{La integral de Choquet como operador de agregación}

\author[Waissman]{Julio Waissman Vilanova}

\institute[UNISON] %
{Departamento de Matem\'aticas\\
 \textbf{Universidad de Sonora}\\
}

\date{}

\subject{Fuzzy Systems}


\begin{document}


\begin{frame}
  \titlepage
\end{frame}

\begin{frame}
  \frametitle{Plan de la presentación}
  \tableofcontents
  %  % You might wish to add the option [pausesections]
\end{frame}


% Sección
\section{Operadores de agregación}
\begin{frame}
  \frametitle{Introducción}
  \begin{itemize}
  \item Los operadores de agregación tienen que ver con \emph{agregar}
    o \emph{fusionar} valores provenientes de diferentes fuentes, para
    obtener un solo valor que los represente. El ejemplo prototipo es
    la media aritmética.  \vskip 6 mm
  \item Herramienta muy importante en la solución de problemas de toma
    de decisión multicriterio (MCDM, por \emph {multi-criteria
      decision making}) en los que, quien toma la decisión (DM, por
    \emph {decision maker}) va a elegir una alternativa entre varias,
    considerando diferentes puntos de vista o criterios.
  \end{itemize}
\end{frame}

\begin{frame}
  \frametitle{Notación}
  Partiendo de un \emph{conjunto de alternativas potenciales}
    $$
    X=X_1\times \ldots \times X_n,
    $$
    cada \emph{alternativa}
    $$
    x:=(x_1, \ldots, x_n)
    $$ 
    \vskip 2 mm
    puede ser descrita como un vector de $n$ componentes, llamadas
    \emph{criterios}.
  \end{frame}
  % DEFINICIÓN
  \begin{frame}
    \frametitle{Operación de agregación n--aria}
    \begin{Def}
      Una \emph{operación de agregación $n$--aria} es una función
	$$
	F^{(n)}:I^n \to I
	$$
	que satisface lo siguiente: \vskip 2 mm
	\begin{itemize}
        \item Monotonía. Esto es, si consideramos dos $n$--tuplas
          $(x_1, \ldots, x_n)$ y $(y_1, \ldots, y_n)$ tal que
          $x_i \le y_i$ para todo $i$, $1 \le i \le n$, se cumple
          $F^{(n)}(x_1, \ldots, x_n ) \le F^{(n)} (y_1, \ldots, y_n)$.
          \vskip 3 mm
        \item Condiciones de frontera de la forma:
		$$
		\inf_{x \in I^n} F^{(n)}(x) = \inf I \mbox{\hspace{7
                    mm} y \hspace{7 mm}} \sup_{x \in I^n} F^{(n)}(x) =
                \sup I
		$$
              \end{itemize}
            \end{Def}
          \end{frame}

          \begin{frame}
            \frametitle{Operador de agregación}
            \begin{Def} % Operador de agregación
              Un \emph{operador de agregación} es una función
	$$
	F: \bigcup_{n \in \mathbb{N}} I^n \to I
	$$ 
	tal que: \vskip 3mm
	\begin{itemize}
	\item para todo $n>1$, $F^{(n)}=F _{|{I^{(n)}}}$ es una
          operación de agregación $n$--aria.  \vskip 3 mm
	\item $F^{(1)}$ es la identidad en I.
	\end{itemize}
      \end{Def}
    \end{frame}

    % Sección OPERADORES DE CONSENSO
    \section{Operadores de consenso}

    \begin{frame}
      \frametitle{Operadores de consenso}
      Las operaciones de consenso son aquellos tales que
  $$
  \min(x_1, \ldots, x_n) \le F(x_1, \ldots, x_n) \le \max(x_1, \ldots,
  x_n).
  $$
  \vskip 3 mm Particularmente interesantes en la toma de decisión
  difusa, al querer simular la agregación de información proveniente
  de diferentes fuentes.
\end{frame}

% SubSecc. WAM
\subsection{Media aritmética ponderada}
  
\begin{frame}
  \frametitle{Media aritmética ponderada}
  \begin{Def}
    La \emph{media aritmética con peso} (WAM, por \emph{weighted
      arithmetic mean}) se define: $M_{\mathbf{w}}$, donde
    $\mathbf{w} = (w_1, \ldots,w_n)$, $w_i > 0$,
    $\sum_{i=1}^n w_i = 1$ y
    $$
    M_{\mathbf{w}} = \sum_{i = 1}^n w_i x_i.
    $$
  \end{Def}
\end{frame}

% SubSecc. OWA
\subsection{Promedio ponderado ordenado}

\begin{frame}
  \frametitle{Promedio ponderado ordenado}
  \begin{Def}
    Un \emph{promedio ponderado ordenado} (OWA, por \emph{ordered weighted
      average}) de dimensión $n$ está dada por
    $$
    M'_{\mathbf{w}} = \sum_{i = 1}^n w_i x_{(i)},
    $$
    \vskip 4 mm donde $(\cdot)$ indica una permutación de
    $\{ 1,\ldots , n\}$, tal que $x_{(1)} \le \ldots \le x_{(n)}$,
    $w_i > 0$, $\sum_{i=1}^n w_i = 1$.
  \end{Def}
\end{frame}

\begin{frame}
  \frametitle{Sobre las WAM y los OWA}
  \begin{itemize}
  \item El OWA puede ser visto como la simetrización de la media
    aritmética ponderada.  \vskip 5 mm
  \item Tanto las WAM como los OWA figuran entre las operación de
    agregación más utilizadas, en la solución de problemas MCDM.
    \vskip 5 mm
  \item La desventaja, muy conocida, es que no siempre es posible
    representar mediante ellas las preferencias del DM.
  \end{itemize}
\end{frame}

% SubSecc. ICh
\subsection{Integral de Choquet}

\begin{frame}
  \frametitle{La integral de Choquet. Motivación}
  \begin{itemize}
  \item La desventaja que presentan la WAM y el OWA se convierten en
    la principal motivación para el estudio de la integral de Choquet.
    \vskip 6 mm
  \item A diferencia de la WAM y el OWA, con la integral de Choquet es
    posible representar las preferencias del DM, cuando existe
    relación entre criterios.
  \end{itemize}
\end{frame}

\begin{frame}
  \frametitle{Medida borrosa}
  \begin{Def}
    Una \emph {medida borrosa} o \emph{capacidad} sobre un conjunto de
    índices $N=\{1,\ldots ,n \}$ es una función monótona
    $\mu :2^N \to [0,1]$, con $\mu(\emptyset)=0$ y $\mu(N)=1$.
  \end{Def}
  \vskip 3 mm
  \begin{itemize}
  \item Se denota como $\mathcal F_N$ al conjunto de todas las medidas
    borrosas sobre $N$.  \vskip 4 mm
  \item Que sea monótona significa que $\mu (S) \le \mu (T)$ si
    $S\subseteq T$.  \vskip 4 mm
  \item $\mu (S)$ puede ser vista como el \emph{peso} que le
    corresponde al subconjunto $S$ de criterios.
  \end{itemize}
\end{frame}

\begin{frame}
  \frametitle{Integral de Choquet}
  \begin{Def}
    Dada $\mu \in \mathcal F_N$, la \emph {integral de Choquet} de
    $x \in I^n$ respecto a $\mu$ se define como \vskip 2 mm
      $$
      \mathcal C_{\mu}(x):=\sum _{i=1}^n x_{(i)}[\mu (A_{(i)})-\mu
      (A_{(i+1)})],
      $$
      \vskip 4 mm donde $A_{(i)}=\{ (i),\ldots ,(n)\}$ y $A_{n+1}=0$.
    \end{Def}
  \end{frame}

  \begin{frame}
    \frametitle{Casos especiales}
    \begin{itemize}
    \item \emph {Propiedad de aditividad}. Una capacidad es aditiva si
      para conjuntos disjuntos $S, T \subseteq N$, se tiene
      $\mu (S \cup T)=\mu (S)+\mu (T)$.  
      \vskip 4 mm 
      \alert{Si $\mu$ es aditiva, la integral de Choquet se reduce a
      una WAM.}  \vskip 8 mm
    \item \emph {Propiedad de simetría}. Una capacidad es simétrica si
      para cualesquiera subconjuntos de índices $S$ y $T$, $|S|=|T|$
      implica $\mu (S)=\mu (T)$.  
      \vskip 4 mm 
      \alert{Si $\mu$
      es simétrica, la integral de Choquet se reduce a un OWA.}
    \end{itemize}
  \end{frame}

  \begin{frame}
    \frametitle{Caracterización de la integral de Choquet}
    Una propiedad fundamental de la integral de Choquet, que resulta
    de la definición, es: \vskip 2 mm
    $$
    \mathcal C_\mu (1_S, 0_{-S})=\mu (S), \hspace {3 mm} \forall S
    \subseteq N.
    $$
    \vskip 3 mm 
    donde, $(1_S, 0_{-S})$ denota una alternativa para la
    que cada característica $x_i$ tomará el valor $1$ si $i \in S$ y
    $0$ en otro caso.
    \vskip 2 mm
    \begin{center}
    \alert{Se caracteriza completamente la integral de Choquet
    al definir su valor en los casos extremos.}
   \end{center}
\end{frame}

  % Sección EJEMPLO ILUSTRATIVO
  \section{Ejemplo ilustrativo}

  \begin{frame}
    \frametitle{Ejemplo ilustrativo}
    Sean $a$, $b$ y $c$ tres alternativas evaluadas sobre tres
    criterios $x_1$, $x_2$ y $x_3$, esquematizados de la siguiente
    forma:
    $$
      \begin{array}{rccc}
        &  x_1 & x_2 & x_3 \\
        a & 0.6 & 0.6 & 1\\
        b & 0.9 & 0 & 1\\
        c & 0 & 0.9 & 1
      \end{array}
     $$
     \vskip 4 mm Se tratará de modelar la preferencia del DM:
     $$
     a \prec b \prec c
     $$
   \end{frame}

   % Modelado con WAM
   \subsection{Modelado con WAM}
   \begin{frame}
     \frametitle{Modelado con la media aritmética ponderada}
     Obtener los pesos adecuados acorde a las preferencias, por
     separado:
        $$
         \begin{tabular}{rc}
           $b \prec c$: & $0.9 w_1+0w_2+1w_3<0w_1+0.9w_2+1w_3$\\
                  & $w_1<w_2$\\
                  % \\
           $a \prec b$: & $0.6w_1+0.6w_2+1w_3<0.9w_1+0w_2+1w_3$\\
                  & $\hspace {8 mm}w_2<0.5w_1$
         \end{tabular}
         $$
         \vskip 2 mm Al tomar en cuenta la preferencia global del DM
         $(a \prec b \prec c)$ tendremos:
         $$
         \begin{tabular}{rl}
           $w_1<w_2<0.5w_1$ & No es posible
         \end{tabular}
         $$
         \vskip 2 mm 
         \alert{La media aritmética ponderada no modela la preferencia del DM.}
       \end{frame}

       % Modelado con OWA
       \subsection{Modelado con OWA}
       \begin{frame}
         \frametitle{Modelado con un promedio ponderado ordenado}
         Obtener los pesos adecuados: \vskip 2 mm
        $$
         \begin{tabular}{rc}
           $b \prec c$: & $0 w_1+0.9w_2+1w_3<0w_1+0.9w_2+1w_3$\\
                  & $0.9w_2+w_3<0.9w_2+w_3$ 
         \end{tabular}
         $$
         \vskip 6 mm 
         \alert{Al usar el OWA las alternativas $b$ y $c$ siempre tendrán el mismo valor.}
       \end{frame}

       % Modelado con ICh
       \subsection{Modelado con la integral de Choquet}
       \begin{frame}
         \frametitle{Modelado con la integral de Choquet}
         Tomando en cuenta que la integral de Choquet se define por
         los vértices del $n$-cubo, lo que se requiere es obtener las
         capacidades (o \emph{pesos}) correspondientes.
         $$
         \begin{tabular}{llll}
           $0$ & $0$ & $0$ & $\mu(\emptyset)=0$, por definición\\
           $1$ & $0$ & $0$ & $\mu(\{1\})$\\
           $0$ & $1$ & $0$ & $\mu(\{2\})$\\ 
           $0$ & $0$ & $1$ & $\mu(\{3\})$\\
           $1$ & $1$ & $0$ & $\mu(\{1, 2\})$\\ 
           $1$ & $0$ & $1$ & $\mu(\{1, 3\})$\\
           $0$ & $1$ & $1$ & $\mu(\{2, 3\})$\\
           $1$ & $1$ & $1$ & $\mu(\{1, 2, 3\})=1$, por definición\\ 
               &  &  & $\mu(\{4, 3\}) = 0$, por definición\\          
         \end{tabular}
         $$
       \end{frame}

       \begin{frame}
         \frametitle{Evaluación de las alternativas}
         Al evaluar las alternativas tenemos:
         \begin{small}
           \begin{itemize}
           \item Alternativa $a$
             \begin{eqnarray*}
               \mathcal{C}(0.6,0.6,1) &=& 0.6(\mu(\{1, 2, 3\}) -
                                          \mu(\{2, 3\})) + \\
                                      & & 0.6(\mu(\{2, 3\}) -
                                          \mu(\{3\})) +  
                                          1 (\mu(\{3\})-\mu(\{4, 3\}))\\
                                      &=& 0.6 + 0.4 \mu(\{3\})
             \end{eqnarray*}
           \item Alternativa $b$
             \begin{eqnarray*}
               \mathcal{C}(0.9,0,1) &=& 0 (\mu(\{1, 2, 3\}) -
                                        \mu(\{1, 3\})) +\\
                                    & & 0.9(\mu(\{1, 3\}) - \mu(\{3\})) +
                                        1 (\mu(\{3\}) - \mu(\{4, 3\}))\\
                                    &=& 0.9 \mu(\{1, 3\}) + 0.1 \mu(\{3\})
             \end{eqnarray*}
           \item Alternativa $c$
             \begin{eqnarray*}
               \mathcal{C}(0,0.9,1) &=& 0 (\mu(\{1, 2, 3\}) - \mu(\{2,
                                        3\})) + \\
                                    & & 0.9 (\mu(\{2, 3\}) -
                                        \mu(\{3\})) + 
                                        1 (\mu(\{3\}) - \mu(\{4, 3\}))\\
                                    &=& 0.9 \mu(\{2, 3\}) + 0.1 \mu(\{3\})
             \end{eqnarray*}
           \end{itemize}
         \end{small}
       \end{frame}

       \begin{frame}
         \frametitle{Evaluación}
         Al evaluar las preferencias parciales se obtiene lo siguiente: 
         \vskip 3 mm
         $$
         \begin{tabular}{rc}
           $b \prec c$: & $0.9 \mu(\{1, 3\}) + 0.1 \mu(\{3\}) < 0.9
                          \mu(\{2, 3\}) + 0.1 \mu(\{3\})$\\
                        & $\mu(\{1, 3\}) < \mu(\{2, 3\})$\\
           \\
           $a \prec b$: & $0.6 + 0.4 \mu(\{3\}) < 0.9 \mu(\{1, 3\}) +
                          0.1 \mu(\{3\})$\\
                        &  $\frac{\mu(\{3\}) + 2}{3} < \mu(\{1, 3\})$
         \end{tabular}
         $$
         \vskip 6 mm 
         Tomando en cuenta la preferencia del DM $(a \prec b \prec c)$ tenemos:
         $$
         \frac{\mu(\{3\}) + 2}{3} < \mu(\{1, 3\}) < \mu(\{2, 3\})
         $$
       \end{frame}

       \begin{frame}
         \frametitle{Asignar valor a las capacidades}
         Asignando un valor a $\mu(\{3\})$, digamos
    $$
    \mu(\{3\}) = 0.4,
    $$
    estamos en condiciones de asignar valores a
    $\mu(\{1, 3\})$ y $\mu(\{2, 3\})$.
    \vskip 4 mm
    Atendiendo a la expresión anterior y al valor asignado a $ \mu(\{3\})$, vemos que $\mu(\{1, 3\})$
    debe ser mayor a $0.8$, pongamos:
    $$
    \mu(\{1, 3\}) = 0.9
    $$
    y, dado que $\mu(\{2, 3\})$ debe ser mayor a $\mu(\{1, 3\})$, podemos asignar:
    $$
    \mu(\{2, 3\}) = 0.95
    $$
    \vskip 4 mm De esta manera, se satisface la preferencia del DM:
    $$
    a \prec b \prec c
    $$
  \end{frame}

  \begin{frame}
  \frametitle{Analisis de los resultados}
  Mediante este modesto ejemplo, es posible resaltar algunos aspectos:
  \begin{itemize}
  \item El que se requiera un valor para la capacidad $\mu(\{1, 3\})$
    significa que existe relación entre los criterios 1 y
    3. Similarmente, por $\mu(\{2, 3\})$ se advierte la relación entre los
    criterios 2 y 3.
  \item La integral de Choquet resultó adecuada, debido a que puede
    representar relaciones entre criterios.
  \item Para satisfacer los requerimientos del DM mediante la integral
    de Choquet, no fueron necesarios los valores de las $2^N=2^3=8$
    capacidades o \emph{pesos}.
  \item Respetando los valores obtenidos para las capacidades
    requeridas no es posible asignar
    valores a las capacidades faltantes, de forma que se dé aditividad
    o simetría. 
  \end{itemize}
\end{frame}

\begin{frame}
  \frametitle{Conclusiones}
  \begin{itemize}
  \item La integral de Choquet es adecuada en la solución de problemas multicriterio,
    cuando existe relación entre criterios.  
    \vskip 6 mm
  \item El problema fundamental radica en la obtención de $2^N$
    capacidades donde $N$ es el número de características. 
    \vskip 6 mm   
  \item Un porcentaje de valores de las capacidades se debe modelar de
    acuerdo a los requerimientos del DM, mientras que otros deben de
    ser asignados de forma que el operador de consenso cumpla con
    propiedades \emph{deseables}.
  \end{itemize}
\end{frame}

% MAS INFORMACIÓN EN:
\begin{frame}
  \frametitle{Más información}

   \begin{thebibliography}{12}

     \beamertemplatearticlebibitems
     % Followed by interesting articles. Keep the list short.
     
   \bibitem{grabisch2010}
     M. Grabisch y C. Labreuche, 
     \newblock \textbf{A decade of application of the Choquet and Sugeno
     integrals in multi-criteria decision aid}, 
     \newblock \emph{Ann. Oper. Res.}, vol. 175, no 1, pp. 247-286, 2010.
   \end{thebibliography}
 \end{frame}

 \begin{frame}
   \frametitle{ }
   \begin{center}
     \Huge{Muchas gracias.}
   \end{center}
 \end{frame}
 
 
\end{document}
